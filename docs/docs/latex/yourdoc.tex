%% Generated by Sphinx.
\def\sphinxdocclass{report}
\documentclass[letterpaper,10pt,english]{sphinxmanual}
\ifdefined\pdfpxdimen
   \let\sphinxpxdimen\pdfpxdimen\else\newdimen\sphinxpxdimen
\fi \sphinxpxdimen=.75bp\relax
\ifdefined\pdfimageresolution
    \pdfimageresolution= \numexpr \dimexpr1in\relax/\sphinxpxdimen\relax
\fi
%% let collapsible pdf bookmarks panel have high depth per default
\PassOptionsToPackage{bookmarksdepth=5}{hyperref}


\PassOptionsToPackage{warn}{textcomp}
\usepackage[utf8]{inputenc}
\ifdefined\DeclareUnicodeCharacter
% support both utf8 and utf8x syntaxes
  \ifdefined\DeclareUnicodeCharacterAsOptional
    \def\sphinxDUC#1{\DeclareUnicodeCharacter{"#1}}
  \else
    \let\sphinxDUC\DeclareUnicodeCharacter
  \fi
  \sphinxDUC{00A0}{\nobreakspace}
  \sphinxDUC{2500}{\sphinxunichar{2500}}
  \sphinxDUC{2502}{\sphinxunichar{2502}}
  \sphinxDUC{2514}{\sphinxunichar{2514}}
  \sphinxDUC{251C}{\sphinxunichar{251C}}
  \sphinxDUC{2572}{\textbackslash}
\fi
\usepackage{cmap}
\usepackage[T1]{fontenc}
\usepackage{amsmath,amssymb,amstext}
\usepackage{babel}



\usepackage{tgtermes}
\usepackage{tgheros}
\renewcommand{\ttdefault}{txtt}



\usepackage[Bjarne]{fncychap}
\usepackage{sphinx}

\fvset{fontsize=auto}
\usepackage{geometry}


% Include hyperref last.
\usepackage{hyperref}
% Fix anchor placement for figures with captions.
\usepackage{hypcap}% it must be loaded after hyperref.
% Set up styles of URL: it should be placed after hyperref.
\urlstyle{same}

\addto\captionsenglish{\renewcommand{\contentsname}{Chapters:}}

\usepackage{sphinxmessages}
\setcounter{tocdepth}{1}



\title{DocName}
\date{Mar 15, 2023}
\release{0.0.1}
\author{YourName}
\newcommand{\sphinxlogo}{\vbox{}}
\renewcommand{\releasename}{Release}
\makeindex
\begin{document}

\ifdefined\shorthandoff
  \ifnum\catcode`\=\string=\active\shorthandoff{=}\fi
  \ifnum\catcode`\"=\active\shorthandoff{"}\fi
\fi

\pagestyle{empty}
\sphinxmaketitle
\pagestyle{plain}
\sphinxtableofcontents
\pagestyle{normal}
\phantomsection\label{\detokenize{index::doc}}


\begin{sphinxadmonition}{note}{Note:}
\sphinxAtStartPar
Anleitung für Tamara zur Zeiterfassung
von dem Zolligerheim Arbeitstuden.
\end{sphinxadmonition}

\sphinxAtStartPar
Hier die Vorbedinungen die erfüllt sein müssen:

\begin{sphinxVerbatim}[commandchars=\\\{\}]
\PYG{n}{PC} \PYG{l+s+s2}{\PYGZdq{}}\PYG{l+s+s2}{fujitsu}\PYG{l+s+s2}{\PYGZdq{}} \PYG{n}{einschalten} \PYG{p}{(}\PYG{n}{power} \PYG{n}{ON}\PYG{p}{)}
\end{sphinxVerbatim}

\sphinxstepscope


\chapter{Intro}
\label{\detokenize{intro:intro}}\label{\detokenize{intro::doc}}
\sphinxAtStartPar
Hier eine kurze Erklärung was
in dem Projekt gemacht wird

\sphinxAtStartPar
This is the first section of your article that introduces your reader to your article and the goals they will accomplish after reading. This section ought to motivate the reader, summarize; what the article is about, why the reader should read, and what the reader will learn/ do in the article. Ensure to make this one to three paragraphs long.

\sphinxAtStartPar
Here is an example from my previous article (Image Optimization and Transformation with Cloudinary):

\sphinxstepscope


\chapter{Mail öffnen}
\label{\detokenize{intro_01:mail-offnen}}\label{\detokenize{intro_01::doc}}
\sphinxAtStartPar
Das Mail öffnen und die Datei speichern unter

\begin{sphinxVerbatim}[commandchars=\\\{\}]
\PYG{n}{Documents}
\end{sphinxVerbatim}

\begin{figure}[htbp]
\centering
\capstart

\noindent\sphinxincludegraphics[scale=0.5]{{fig_01}.png}
\caption{Image caption}\label{\detokenize{intro_01:id1}}\end{figure}

\sphinxAtStartPar
Das Verzeichniss öffnen

\begin{sphinxVerbatim}[commandchars=\\\{\}]
\PYG{n}{Zollingerheim} \PYG{n}{Tierpflege} \PYG{n}{Arbeitstunden}
\end{sphinxVerbatim}

\sphinxAtStartPar
danach

\begin{sphinxVerbatim}[commandchars=\\\{\}]
\PYG{n}{speichern}
\end{sphinxVerbatim}

\sphinxstepscope


\chapter{Datei abspeichern}
\label{\detokenize{intro_02:datei-abspeichern}}\label{\detokenize{intro_02::doc}}
\sphinxAtStartPar
Zeitangabe mit : z.b. 17:30
alles kontrollieren

\begin{sphinxVerbatim}[commandchars=\\\{\}]
\PYG{n}{Datei} \PYG{n}{seichern} \PYG{n}{unter}\PYG{o}{.}\PYG{o}{.}\PYG{o}{.}\PYG{o}{.}
\end{sphinxVerbatim}

\begin{sphinxadmonition}{tip}{Tip:}
\sphinxAtStartPar
Möchten sie diese ersetzen =ja wählen
\end{sphinxadmonition}

\sphinxAtStartPar
abbrechen
Datei Libre Office beenden

\begin{sphinxVerbatim}[commandchars=\\\{\}]
\PYG{n}{Nun} \PYG{n}{sollte} \PYG{n}{ein} \PYG{n}{blauer} \PYG{n}{Bildschirm} \PYG{n}{sichtbar} \PYG{n}{sein}\PYG{o}{.}
\end{sphinxVerbatim}

\sphinxstepscope


\chapter{Arbeitsstunden erfassen}
\label{\detokenize{intro_03:arbeitsstunden-erfassen}}\label{\detokenize{intro_03::doc}}
\sphinxAtStartPar
\sphinxcode{\sphinxupquote{Windows}} \sphinxcode{\sphinxupquote{Taste}} drücken
auf der Tastatur

\begin{sphinxVerbatim}[commandchars=\\\{\}]
\PYG{n}{Arbeitsstunden} \PYG{n}{schreiben} \PYG{n}{und} \PYG{n}{Enter} \PYG{n}{drücken}\PYG{o}{.}
\end{sphinxVerbatim}

\begin{sphinxadmonition}{tip}{Tip:}
\sphinxAtStartPar
Datei mit dem richtigen Jahr und Monat wählen.
\end{sphinxadmonition}

\sphinxstepscope


\chapter{Datei wählen}
\label{\detokenize{intro_04:datei-wahlen}}\label{\detokenize{intro_04::doc}}
\sphinxAtStartPar
Bei Name nach Documents \sphinxcode{\sphinxupquote{ZH Tierpflege}} suchen

\begin{sphinxVerbatim}[commandchars=\\\{\}]
\PYG{n}{Doppelclick} \PYG{n}{auf} \PYG{n}{Datei}
\end{sphinxVerbatim}

\sphinxstepscope


\chapter{Mail senden}
\label{\detokenize{intro_05:mail-senden}}\label{\detokenize{intro_05::doc}}
\sphinxAtStartPar
Auswählen

\begin{sphinxVerbatim}[commandchars=\\\{\}]
\PYG{n}{Datei} \PYG{n}{wählen} \PYG{o}{=} \PYG{n}{Senden}
\PYG{n}{Dokument} \PYG{n}{als} \PYG{n}{email}
\end{sphinxVerbatim}

\sphinxAtStartPar
Email eintragen

\begin{sphinxVerbatim}[commandchars=\\\{\}]
\PYG{n}{martina}\PYG{o}{.}\PYG{n}{wehrli}\PYG{n+nd}{@zolliger}\PYG{o}{\PYGZhy{}}\PYG{n}{stiftung}\PYG{o}{.}\PYG{n}{ch}
\end{sphinxVerbatim}

\begin{sphinxadmonition}{attention}{Attention:}
\sphinxAtStartPar
Mit x kann wenn notwendig gelöscht werden
\end{sphinxadmonition}

\sphinxAtStartPar
Auf weisser Fläche \sphinxcode{\sphinxupquote{text}} schreiben

\begin{sphinxVerbatim}[commandchars=\\\{\}]
\PYG{n}{Liebe} \PYG{n}{Martina}\PYG{p}{,}

\PYG{n}{Schicke} \PYG{n}{Dir} \PYG{n}{noch} \PYG{n}{meine} \PYG{n}{Arbeitsstunden}
\PYG{n}{abrechnung} \PYG{n}{von} \PYG{n}{dem} \PYG{n}{Oktober}

\PYG{n}{Liebe} \PYG{n}{Grüsse} \PYG{n}{Tamara}
\end{sphinxVerbatim}

\sphinxAtStartPar
\sphinxstylestrong{Mail versenden fertig}

\sphinxstepscope


\chapter{Installation}
\label{\detokenize{install:installation}}\label{\detokenize{install::doc}}
\sphinxAtStartPar
At the command line:

\sphinxAtStartPar
easy\_install crawler

\sphinxAtStartPar
Or, if you have pip installed:

\sphinxAtStartPar
pip install crawler
\begin{quote}

\sphinxAtStartPar
Hier kan irgendein Text stehen.
Das kann auch eine sehr lange Textzeile sein die eigentlich zu lang zum lesen ist.
\end{quote}

\begin{DUlineblock}{0em}
\item[] so
\item[] kann
\item[] man
\item[] Zeilenumbrüche erstellen.
\end{DUlineblock}
\begin{description}
\sphinxlineitem{Bullet List:}\begin{itemize}
\item {} 
\sphinxAtStartPar
Chapter 1
\begin{itemize}
\item {} 
\sphinxAtStartPar
Unterkapitel 1

\item {} 
\sphinxAtStartPar
Unterkapitel 2

\end{itemize}

\end{itemize}

\end{description}

\begin{sphinxVerbatim}[commandchars=\\\{\},numbers=left,firstnumber=1,stepnumber=1]
\PYG{k+kn}{from} \PYG{n+nn}{microbit} \PYG{k+kn}{import} \PYG{o}{*}

\PYG{n}{x} \PYG{o}{=} \PYG{l+m+mi}{0}
\PYG{k}{for} \PYG{n}{y} \PYG{o+ow}{in} \PYG{n+nb}{range}\PYG{p}{(}\PYG{l+m+mi}{0}\PYG{p}{,} \PYG{l+m+mi}{5}\PYG{p}{)}\PYG{p}{:}
  \PYG{n}{display}\PYG{o}{.}\PYG{n}{set\PYGZus{}pixel}\PYG{p}{(}\PYG{n}{x}\PYG{p}{,} \PYG{n}{y}\PYG{p}{,} \PYG{l+m+mi}{9}\PYG{p}{)}
\end{sphinxVerbatim}

\sphinxAtStartPar
Und hier nun eine url:
.. \_http://www.ibm.com
\begin{quote}

\sphinxAtStartPar
So kann man z.b besonderer Text hervorheben: \sphinxcode{\sphinxupquote{soises}} würde der Deutsche Bundekanzler Anwärter
der in Baden\sphinxhyphen{}Würtenberg Minister Präsitent gewesen ist antworten. Eine \sphinxhref{http://sphinx-doc.org}{A cool website} sphinx.
\begin{quote}

\sphinxAtStartPar
A cool bit of code:

\begin{sphinxVerbatim}[commandchars=\\\{\}]
\PYG{n}{Hier} \PYG{n}{steht} \PYG{n}{dann} \PYG{n}{irgendetwas}
\end{sphinxVerbatim}

\begin{sphinxVerbatim}[commandchars=\\\{\}]
A bit of \PYG{g+gs}{**rst**} which should be \PYG{g+ge}{*highlighted*} properly.
\end{sphinxVerbatim}
\end{quote}
\end{quote}

\sphinxAtStartPar
The ‘rm’ command is very dangerous.  If you are logged
in as root and enter

\begin{sphinxVerbatim}[commandchars=\\\{\}]
\PYG{n}{cd} \PYG{o}{/}
\PYG{n}{rm} \PYG{o}{\PYGZhy{}}\PYG{n}{rf} \PYG{o}{*}
\end{sphinxVerbatim}
\noindent
you will erase the entire contents of your file system.

\begin{sphinxadmonition}{note}{Note:}
\begin{DUlineblock}{0em}
\item[] Beware ot the cat !
\item[] or the Dog ?
\end{DUlineblock}
\end{sphinxadmonition}

\begin{sphinxShadowBox}
\sphinxstyletopictitle{Machen wir mal Inhalt}
\begin{itemize}
\item {} 
\sphinxAtStartPar
\phantomsection\label{\detokenize{install:id8}}{\hyperref[\detokenize{install:installation}]{\sphinxcrossref{Installation}}}
\begin{itemize}
\item {} 
\sphinxAtStartPar
\phantomsection\label{\detokenize{install:id9}}{\hyperref[\detokenize{install:section-title}]{\sphinxcrossref{Section Title}}}
\begin{itemize}
\item {} 
\sphinxAtStartPar
\phantomsection\label{\detokenize{install:id10}}{\hyperref[\detokenize{install:section-title-wieso}]{\sphinxcrossref{\sphinxstyleemphasis{Section Title Wieso}}}}

\item {} 
\sphinxAtStartPar
\phantomsection\label{\detokenize{install:id11}}{\hyperref[\detokenize{install:section-2-title-wieso}]{\sphinxcrossref{\sphinxhyphen{}Section 2 Title Wieso\sphinxhyphen{}}}}

\item {} 
\sphinxAtStartPar
\phantomsection\label{\detokenize{install:id12}}{\hyperref[\detokenize{install:section-3-title-wieso}]{\sphinxcrossref{Section 3 Title Wieso}}}
\begin{itemize}
\item {} 
\sphinxAtStartPar
\phantomsection\label{\detokenize{install:id13}}{\hyperref[\detokenize{install:id2}]{\sphinxcrossref{Section Title}}}

\item {} 
\sphinxAtStartPar
\phantomsection\label{\detokenize{install:id14}}{\hyperref[\detokenize{install:id3}]{\sphinxcrossref{Section Title}}}

\item {} 
\sphinxAtStartPar
\phantomsection\label{\detokenize{install:id15}}{\hyperref[\detokenize{install:id4}]{\sphinxcrossref{Section Title}}}
\begin{itemize}
\item {} 
\sphinxAtStartPar
\phantomsection\label{\detokenize{install:id16}}{\hyperref[\detokenize{install:id5}]{\sphinxcrossref{Section Title}}}

\item {} 
\sphinxAtStartPar
\phantomsection\label{\detokenize{install:id17}}{\hyperref[\detokenize{install:id6}]{\sphinxcrossref{Section Title}}}

\item {} 
\sphinxAtStartPar
\phantomsection\label{\detokenize{install:id18}}{\hyperref[\detokenize{install:id7}]{\sphinxcrossref{Section Title}}}

\end{itemize}

\end{itemize}

\end{itemize}

\end{itemize}

\end{itemize}
\end{sphinxShadowBox}


\section{Section Title}
\label{\detokenize{install:section-title}}

\subsection{\sphinxstyleemphasis{Section Title Wieso}}
\label{\detokenize{install:section-title-wieso}}

\subsection{\sphinxhyphen{}Section 2 Title Wieso\sphinxhyphen{}}
\label{\detokenize{install:section-2-title-wieso}}

\subsection{Section 3 Title Wieso}
\label{\detokenize{install:section-3-title-wieso}}

\subsubsection{Section Title}
\label{\detokenize{install:id2}}

\subsubsection{Section Title}
\label{\detokenize{install:id3}}

\subsubsection{Section Title}
\label{\detokenize{install:id4}}

\paragraph{Section Title}
\label{\detokenize{install:id5}}

\paragraph{Section Title}
\label{\detokenize{install:id6}}

\paragraph{Section Title}
\label{\detokenize{install:id7}}
\sphinxstepscope


\chapter{Support}
\label{\detokenize{support:support}}\label{\detokenize{support::doc}}

\chapter{Indices and tables}
\label{\detokenize{index:indices-and-tables}}\begin{itemize}
\item {} 
\sphinxAtStartPar
\DUrole{xref,std,std-ref}{genindex}

\item {} 
\sphinxAtStartPar
\DUrole{xref,std,std-ref}{modindex}

\item {} 
\sphinxAtStartPar
\DUrole{xref,std,std-ref}{search}

\end{itemize}



\renewcommand{\indexname}{Index}
\printindex
\end{document}